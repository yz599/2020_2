\title{CS 615 - Deep Learning}
\author{
        Assignment 3 - Convolutional Neural Networks\\
        Spring 2020\\
        Xiangang Lai
}
\date{}
\documentclass[12pt]{article}
\usepackage[margin=0.7in]{geometry}
\usepackage{graphicx}
\usepackage{float}
\usepackage{amsmath}


\begin{document}
\maketitle

\section{Theory}
\begin{enumerate}
\item (2pts) Apply kernel $K$ to data $X$.  In other words, what is $X*K$?:\\
$$X=\begin{bmatrix}
1&2&3\\
4&5&6\\
7 & 8 & 9
\end{bmatrix}
K=\begin{bmatrix}
1&2&3\\
4&5&6\\
7 & 8 & 9
\end{bmatrix}
$$
Its only valid value at the center: 165

\item Given the feature map filter output, $F=\begin{bmatrix}
1&2&3&4&1&3\\
4&5&6&0&0&12\\
7 & 8 & 9 & 1 & 0 & 4\\
-100 & -100 & -100 & -100 & -100 & -100
\end{bmatrix}$, what is the output from a pooling layer with width of 2 and stride of 2 if we are using:
	\begin{itemize}
		\item (3pts) Max-Pooling?\\
		$P=\begin{bmatrix}
           5&6&12\\
		8&9&4\
		\end{bmatrix}$
		\item (3pts) Mean-Pooling?\\
		$P=\begin{bmatrix}
           3&\frac{13}{4}&4\\
		-\frac{185}{4}&-\frac{95}{2}&-49\
		\end{bmatrix}$
	\end{itemize}
\item (2pts) Given an image $X$, what would the kernel $K$ be that can reproduce the image when convoluted with it?  That is, what is $K$ such that $X*K=X$?\\
K should be the identity matrix with dimension 1
\end{enumerate}

\newpage
\section{CNN for LSE Classification}\label{LSE}

\paragraph{What you will need for your report}
\begin{enumerate}
\item Your hyperparameter choices.
\begin{itemize}
\item[]K is generated randomly with integers from -10 to 10, by seed 2278965
\item[]learning rate: 0.0027
\item[]L2 regularization rate: 0.01
\end{itemize}

\item Image representations of your initial and final kernels.

\begin{figure}[H]
\centering
\includegraphics[width=1\textwidth]{Q2_K1}
\caption{Initial K}
\end{figure} 

\begin{figure}[H]
\centering
\includegraphics[width=1\textwidth]{Q2_K2}
\caption{Final K}
\end{figure} 

\item A plot of the RMSE as a function of the number of iterations.
\begin{figure}[H]
\centering
\includegraphics[width=1\textwidth]{Q2}
\caption{Square error over number of iterations}
\end{figure} 


\end{enumerate}

\newpage
\section{CNN for MLE Classification}
\begin{enumerate}
\item Your hyperparameter choices.
\begin{itemize}
\item[]K is generated randomly with integers from -10 to 10, by seed 28
\item[]learning rate: 0.0006
\item[]L2 regularization rate: 0.001
\end{itemize}

\item Image representations of your initial and final kernels.

\begin{figure}[H]
\centering
\includegraphics[width=1\textwidth]{Q3_K1}
\caption{Initial K}
\end{figure} 

\begin{figure}[H]
\centering
\includegraphics[width=1\textwidth]{Q3_K2}
\caption{Final K}
\end{figure} 

\item A plot of the RMSE as a function of the number of iterations.
\begin{figure}[H]
\centering
\includegraphics[width=1\textwidth]{Q3}
\caption{Log likelihood over number of iterations}
\end{figure} 


\end{enumerate}

\newpage
\section{CNN for LCE Classification}\label{LCE}
\begin{enumerate}
\item Your hyperparameter choices.
\begin{itemize}
\item[]K is generated randomly with integers from -10 to 10, by seed 1153
\item[]learning rate: 0.01
\item[]L2 regularization rate: 0.0001
\end{itemize}

\item Image representations of your initial and final kernels.

\begin{figure}[H]
\centering
\includegraphics[width=1\textwidth]{Q4_K1}
\caption{Initial K}
\end{figure} 

\begin{figure}[H]
\centering
\includegraphics[width=1\textwidth]{Q4_K2}
\caption{Final K}
\end{figure} 

\item A plot of the RMSE as a function of the number of iterations.
\begin{figure}[H]
\centering
\includegraphics[width=1\textwidth]{Q4}
\caption{Cross entropy loss over number of iterations}
\end{figure} 


\end{enumerate}

\newpage
\section{CNN With Multiple Kernels}\label{MF}
\begin{enumerate}
\item Your hyperparameter choices.
\begin{itemize}
\item[]K is generated randomly with integers from -10 to 10, by seed 32
\item[]learning rate: 0.01
\item[]L2 regularization rate: 0
\end{itemize}

\item Image representations of your initial and final kernels.

\begin{figure}[H]
\centering
\includegraphics[width=1\textwidth]{Q5_K1}
\caption{Initial K}
\end{figure} 

\begin{figure}[H]
\centering
\includegraphics[width=1\textwidth]{Q5_K2}
\caption{Final K}
\end{figure} 

\item A plot of the RMSE as a function of the number of iterations.
\begin{figure}[H]
\centering
\includegraphics[width=1\textwidth]{Q5}
\caption{Square error over number of iterations}
\end{figure} 


\end{enumerate}
\newpage
\section{Multi-Kernel CNN For Image Classification}
\paragraph{Architecture 1}
\begin{enumerate}
\item The architecture.
\begin{itemize}
\item[]Activation function: linear
\item[]Objective function: square error
\item[]stride: 2
\item[]Width of kernels: 5
\item[]number of kernels: 4
\end{itemize}

\item The values of any additional hyperparameters that you chose (outside of the architecture decisions).
\begin{itemize}
\item[]K is generated randomly with integers from -10 to 10, by seed 32
\item[]learning rate: 0.00007
\item[]L2 regularization rate: 0.001
\end{itemize}

\item Plot of iteration vs objective function evaluation.
\begin{figure}[H]
\centering
\includegraphics[width=1\textwidth]{Q61}
\caption{Square error over number of iterations}
\end{figure} 
\item Final training and testing accuracies.\\
train accuracy: 1\\
trst accuracy: 0.93

\end{enumerate}


\paragraph{Architecture 2}
\begin{enumerate}
\item The architecture.
\begin{itemize}
\item[]Activation function: logistic
\item[]Objective function: log likelihood
\item[]stride: 2
\item[]Width of kernels: 19
\item[]number of kernels: 9
\end{itemize}

\item The values of any additional hyperparameters that you chose (outside of the architecture decisions).
\begin{itemize}
\item[]K is generated randomly with integers from -10 to 10, by seed 354
\item[]learning rate: 3e-7
\item[]L2 regularization rate: 0.5
\end{itemize}

\item Plot of iteration vs objective function evaluation.
\begin{figure}[H]
\centering
\includegraphics[width=1\textwidth]{Q62}
\caption{Square error over number of iterations}
\end{figure} 
\item Final training and testing accuracies.\\
train accuracy: 0.08\\
trst accuracy: 0.07
\end{enumerate}

\end{document}

